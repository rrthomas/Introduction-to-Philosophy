%
% Formative essay 1
%
% Reuben Thomas
%


\documentclass[english]{scrartcl}
\usepackage{babel,newpxtext,url}
\usepackage[utf8x]{inputenc}



% Alter some default parameters for general typesetting

\frenchspacing
%TC:macro \subtitle [header]


% New commands



\begin{document}

\title{How does induction differ from deduction?\\Why is induction important to reasoning\\in science?}
\subtitle{Formative Essay 1\\%
% Formative essay 1
%
% Reuben Thomas
%


\documentclass[english]{scrartcl}
\usepackage{babel,newpxtext,url}
\usepackage[utf8x]{inputenc}



% Alter some default parameters for general typesetting

\frenchspacing
%TC:macro \subtitle [header]


% New commands



\begin{document}

\title{How does induction differ from deduction?\\Why is induction important to reasoning\\in science?}
\subtitle{Formative Essay 1\\%
% Formative essay 1
%
% Reuben Thomas
%


\documentclass[english]{scrartcl}
\usepackage{babel,newpxtext,url}
\usepackage[utf8x]{inputenc}



% Alter some default parameters for general typesetting

\frenchspacing
%TC:macro \subtitle [header]


% New commands



\begin{document}

\title{How does induction differ from deduction?\\Why is induction important to reasoning\\in science?}
\subtitle{Formative Essay 1\\%
% Formative essay 1
%
% Reuben Thomas
%


\documentclass[english]{scrartcl}
\usepackage{babel,newpxtext,url}
\usepackage[utf8x]{inputenc}



% Alter some default parameters for general typesetting

\frenchspacing
%TC:macro \subtitle [header]


% New commands



\begin{document}

\title{How does induction differ from deduction?\\Why is induction important to reasoning\\in science?}
\subtitle{Formative Essay 1\\\input{Formative essay 1.wc}words}
\author{Reuben Thomas}
\date{21st October 2021}
\maketitle

Inductive inference is generally held to be an entirely distinct mode of reasoning from mathematical induction. In this essay I shall show that, on the contrary, they are structurally similar, and use this similarity to illuminate the importance of inductive inference to science.

Induction and deduction are methods of argument that use inference to reason from premisses to conclusion.

Deduction applies the rules of some logic to the premisses to reach the conclusion: the reasoning is tautological. In an inductive inference, the conclusion is merely supported by the premisses. Deduction is universally accepted as a valid mode of reasoning.

Induction’s importance to science is that it permits generalization: without it, we would have only knowledge by acquaintance and propositional knowledge, and therefore only an omniscient being could work out general laws. Induction is widely accepted not to be a valid mode of reasoning; the classic statement of “the problem of induction” is given by Hume~\cite{hume1978}. For example, I may observe many white swans, and conclude that all swans are white; but Hume says “the sheer number of impressions has in this case no more effect than if we confined ourselves to one only”: that is, concluding that all swans are white from seeing one white swan is just as valid as reaching the conclusion from seeing a hundred. (And Popper might remind us that it only takes a single black swan to falsify the proposition.)

It is instructive to compare inductive reasoning with mathematical induction. These are typically treated as unrelated modes of reasoning that unfortunately share a name: for example, Wikipedia’s articles on the topics~\cite{wiki:mathematical-induction,wiki:inductive-reasoning} both warn the reader not to confuse the one concept with the other. It is true that mathematical induction is a type of deductive reasoning, but, as I will explain, it is nonetheless a form of induction, and the difference between mathematical induction and inductive reasoning is more subtle than it first appears, and illuminates the importance of inductive reasoning to science.

Mathematical induction is a proof technique used to prove that some proposition $P(n)$ holds for every natural number $0, 1,…$. Mathematical induction can be generalized to structures other than the natural numbers; in general, “structural induction” can be used on any well-founded relation; further details are beyond the scope of this essay. A “proof by induction” consists of two cases: the \textbf{base case}, which proves $P(0)$, and the \textbf{induction step}, which proves that $P(n)$ implies $P(n+1)$. A classic example is the proof that the sum of the first $n$ natural numbers is $\frac{n(n+1)}{2}$.

So, we must prove, for all $n$, $P(n): 0+1+…n = \frac{n(n+1)}{2}$.

First, the base case: $P(0)$ is the proposition $0 = \frac{0(0+1)}{2}$. We can simplify the right-hand side to $\frac{0×1}{2}$, which is $\frac{0}{2}$, which is $0$. So, we have proved the base case.

The induction step is to prove that $P(n)\supset P(n+1)$. In other words, we can assume $P(n)$, and we must prove $P(n+1)$. Substituting $n+1$ for $n$ in the definition of $P(n)$, we see that $P(n+1)$ is the proposition $0+1+…n+(n+1) = \frac{(n+1)(n+2)}{2}$. We can use our assumption of $P(n)$ to substitute $\frac{n(n+1)}{2}$ into the left-hand side, obtaining: $\frac{n(n+1)}{2}+(n+1) = \frac{(n+1)(n+2)}{2}$. Now, we multiply out the right-hand bracket of the right-hand side: $\frac{n(n+1)}{2}+(n+1) = \frac{(n+1)n}{2} + \frac{(n+1)2}{2}$. By rearranging the left-hand fraction and cancelling the $2$s in the right-hand fraction, we obtain: $\frac{n(n+1)}{2}+(n+1) = \frac{n(n+1)}{2} + (n+1)$. We can see that the left-hand side and right-hand side are now identical, so we have proved $P(n+1)$. Hence, “by induction”, we have proved $P(n)$ for all $n$.

To see how this works for all $n$, go back to $P(0)$, which we proved above. In the induction step, we showed that if $P(0)$ is true, then $P(1)$ is true. We can now repeat the induction step: assuming $P(1)$ is true, $P(2)$ must be true; and we can continue to prove $P(n)$ for any $n$, because for any desired $n$ we will reach it eventually by counting up from $0$.

The mathematical argument appears at first to be a finite argument: the proofs of $P(0)$ and of $P(n)\supset P(n+1)$ both consist of a finite number of steps. But as we just saw, the proof technique is justified by the structure of the natural numbers: we know that we can reach any given number eventually, however large, by counting. The “glue” of mathematical induction is the logical connection between the numbers, of which we have \emph{a priori} knowledge.

In the physical world, we do not usually claim we have such knowledge; instead, we rely on the assumption that nature is uniform. The sun has risen every day for as far back as we can remember, so we infer by induction that it will rise tomorrow. One electron behaves like another in any experiment we care to make, so we infer by induction that Ohm’s Law will always hold. But our “induction step”, in the case of science, relies on this assumption of uniformity: there is no “well-founded structure” in the physical world on which we can perform mathematical induction.

In mathematics we can be certain of our results; in physics, if we want to formulate general laws, we are forced to be pragmatic, and take the uniformity of nature on trust. But we have a considerable body of evidence on which to rest our confidence: centuries of careful experimentation combine with millennia of everyday observation to convince us that the universe is, indeed, uniform. Induction is not just, as in Reichenbach’s words, “the best wager we can lay”~\cite{reichenbach1938eap}, it is a very good wager indeed.

\bibliographystyle{plain}
\bibliography{philosophy}

\end{document}

% LocalWords:  Reichenbach’s
words}
\author{Reuben Thomas}
\date{21st October 2021}
\maketitle

Inductive inference is generally held to be an entirely distinct mode of reasoning from mathematical induction. In this essay I shall show that, on the contrary, they are structurally similar, and use this similarity to illuminate the importance of inductive inference to science.

Induction and deduction are methods of argument that use inference to reason from premisses to conclusion.

Deduction applies the rules of some logic to the premisses to reach the conclusion: the reasoning is tautological. In an inductive inference, the conclusion is merely supported by the premisses. Deduction is universally accepted as a valid mode of reasoning.

Induction’s importance to science is that it permits generalization: without it, we would have only knowledge by acquaintance and propositional knowledge, and therefore only an omniscient being could work out general laws. Induction is widely accepted not to be a valid mode of reasoning; the classic statement of “the problem of induction” is given by Hume~\cite{hume1978}. For example, I may observe many white swans, and conclude that all swans are white; but Hume says “the sheer number of impressions has in this case no more effect than if we confined ourselves to one only”: that is, concluding that all swans are white from seeing one white swan is just as valid as reaching the conclusion from seeing a hundred. (And Popper might remind us that it only takes a single black swan to falsify the proposition.)

It is instructive to compare inductive reasoning with mathematical induction. These are typically treated as unrelated modes of reasoning that unfortunately share a name: for example, Wikipedia’s articles on the topics~\cite{wiki:mathematical-induction,wiki:inductive-reasoning} both warn the reader not to confuse the one concept with the other. It is true that mathematical induction is a type of deductive reasoning, but, as I will explain, it is nonetheless a form of induction, and the difference between mathematical induction and inductive reasoning is more subtle than it first appears, and illuminates the importance of inductive reasoning to science.

Mathematical induction is a proof technique used to prove that some proposition $P(n)$ holds for every natural number $0, 1,…$. Mathematical induction can be generalized to structures other than the natural numbers; in general, “structural induction” can be used on any well-founded relation; further details are beyond the scope of this essay. A “proof by induction” consists of two cases: the \textbf{base case}, which proves $P(0)$, and the \textbf{induction step}, which proves that $P(n)$ implies $P(n+1)$. A classic example is the proof that the sum of the first $n$ natural numbers is $\frac{n(n+1)}{2}$.

So, we must prove, for all $n$, $P(n): 0+1+…n = \frac{n(n+1)}{2}$.

First, the base case: $P(0)$ is the proposition $0 = \frac{0(0+1)}{2}$. We can simplify the right-hand side to $\frac{0×1}{2}$, which is $\frac{0}{2}$, which is $0$. So, we have proved the base case.

The induction step is to prove that $P(n)\supset P(n+1)$. In other words, we can assume $P(n)$, and we must prove $P(n+1)$. Substituting $n+1$ for $n$ in the definition of $P(n)$, we see that $P(n+1)$ is the proposition $0+1+…n+(n+1) = \frac{(n+1)(n+2)}{2}$. We can use our assumption of $P(n)$ to substitute $\frac{n(n+1)}{2}$ into the left-hand side, obtaining: $\frac{n(n+1)}{2}+(n+1) = \frac{(n+1)(n+2)}{2}$. Now, we multiply out the right-hand bracket of the right-hand side: $\frac{n(n+1)}{2}+(n+1) = \frac{(n+1)n}{2} + \frac{(n+1)2}{2}$. By rearranging the left-hand fraction and cancelling the $2$s in the right-hand fraction, we obtain: $\frac{n(n+1)}{2}+(n+1) = \frac{n(n+1)}{2} + (n+1)$. We can see that the left-hand side and right-hand side are now identical, so we have proved $P(n+1)$. Hence, “by induction”, we have proved $P(n)$ for all $n$.

To see how this works for all $n$, go back to $P(0)$, which we proved above. In the induction step, we showed that if $P(0)$ is true, then $P(1)$ is true. We can now repeat the induction step: assuming $P(1)$ is true, $P(2)$ must be true; and we can continue to prove $P(n)$ for any $n$, because for any desired $n$ we will reach it eventually by counting up from $0$.

The mathematical argument appears at first to be a finite argument: the proofs of $P(0)$ and of $P(n)\supset P(n+1)$ both consist of a finite number of steps. But as we just saw, the proof technique is justified by the structure of the natural numbers: we know that we can reach any given number eventually, however large, by counting. The “glue” of mathematical induction is the logical connection between the numbers, of which we have \emph{a priori} knowledge.

In the physical world, we do not usually claim we have such knowledge; instead, we rely on the assumption that nature is uniform. The sun has risen every day for as far back as we can remember, so we infer by induction that it will rise tomorrow. One electron behaves like another in any experiment we care to make, so we infer by induction that Ohm’s Law will always hold. But our “induction step”, in the case of science, relies on this assumption of uniformity: there is no “well-founded structure” in the physical world on which we can perform mathematical induction.

In mathematics we can be certain of our results; in physics, if we want to formulate general laws, we are forced to be pragmatic, and take the uniformity of nature on trust. But we have a considerable body of evidence on which to rest our confidence: centuries of careful experimentation combine with millennia of everyday observation to convince us that the universe is, indeed, uniform. Induction is not just, as in Reichenbach’s words, “the best wager we can lay”~\cite{reichenbach1938eap}, it is a very good wager indeed.

\bibliographystyle{plain}
\bibliography{philosophy}

\end{document}

% LocalWords:  Reichenbach’s
words}
\author{Reuben Thomas}
\date{21st October 2021}
\maketitle

Inductive inference is generally held to be an entirely distinct mode of reasoning from mathematical induction. In this essay I shall show that, on the contrary, they are structurally similar, and use this similarity to illuminate the importance of inductive inference to science.

Induction and deduction are methods of argument that use inference to reason from premisses to conclusion.

Deduction applies the rules of some logic to the premisses to reach the conclusion: the reasoning is tautological. In an inductive inference, the conclusion is merely supported by the premisses. Deduction is universally accepted as a valid mode of reasoning.

Induction’s importance to science is that it permits generalization: without it, we would have only knowledge by acquaintance and propositional knowledge, and therefore only an omniscient being could work out general laws. Induction is widely accepted not to be a valid mode of reasoning; the classic statement of “the problem of induction” is given by Hume~\cite{hume1978}. For example, I may observe many white swans, and conclude that all swans are white; but Hume says “the sheer number of impressions has in this case no more effect than if we confined ourselves to one only”: that is, concluding that all swans are white from seeing one white swan is just as valid as reaching the conclusion from seeing a hundred. (And Popper might remind us that it only takes a single black swan to falsify the proposition.)

It is instructive to compare inductive reasoning with mathematical induction. These are typically treated as unrelated modes of reasoning that unfortunately share a name: for example, Wikipedia’s articles on the topics~\cite{wiki:mathematical-induction,wiki:inductive-reasoning} both warn the reader not to confuse the one concept with the other. It is true that mathematical induction is a type of deductive reasoning, but, as I will explain, it is nonetheless a form of induction, and the difference between mathematical induction and inductive reasoning is more subtle than it first appears, and illuminates the importance of inductive reasoning to science.

Mathematical induction is a proof technique used to prove that some proposition $P(n)$ holds for every natural number $0, 1,…$. Mathematical induction can be generalized to structures other than the natural numbers; in general, “structural induction” can be used on any well-founded relation; further details are beyond the scope of this essay. A “proof by induction” consists of two cases: the \textbf{base case}, which proves $P(0)$, and the \textbf{induction step}, which proves that $P(n)$ implies $P(n+1)$. A classic example is the proof that the sum of the first $n$ natural numbers is $\frac{n(n+1)}{2}$.

So, we must prove, for all $n$, $P(n): 0+1+…n = \frac{n(n+1)}{2}$.

First, the base case: $P(0)$ is the proposition $0 = \frac{0(0+1)}{2}$. We can simplify the right-hand side to $\frac{0×1}{2}$, which is $\frac{0}{2}$, which is $0$. So, we have proved the base case.

The induction step is to prove that $P(n)\supset P(n+1)$. In other words, we can assume $P(n)$, and we must prove $P(n+1)$. Substituting $n+1$ for $n$ in the definition of $P(n)$, we see that $P(n+1)$ is the proposition $0+1+…n+(n+1) = \frac{(n+1)(n+2)}{2}$. We can use our assumption of $P(n)$ to substitute $\frac{n(n+1)}{2}$ into the left-hand side, obtaining: $\frac{n(n+1)}{2}+(n+1) = \frac{(n+1)(n+2)}{2}$. Now, we multiply out the right-hand bracket of the right-hand side: $\frac{n(n+1)}{2}+(n+1) = \frac{(n+1)n}{2} + \frac{(n+1)2}{2}$. By rearranging the left-hand fraction and cancelling the $2$s in the right-hand fraction, we obtain: $\frac{n(n+1)}{2}+(n+1) = \frac{n(n+1)}{2} + (n+1)$. We can see that the left-hand side and right-hand side are now identical, so we have proved $P(n+1)$. Hence, “by induction”, we have proved $P(n)$ for all $n$.

To see how this works for all $n$, go back to $P(0)$, which we proved above. In the induction step, we showed that if $P(0)$ is true, then $P(1)$ is true. We can now repeat the induction step: assuming $P(1)$ is true, $P(2)$ must be true; and we can continue to prove $P(n)$ for any $n$, because for any desired $n$ we will reach it eventually by counting up from $0$.

The mathematical argument appears at first to be a finite argument: the proofs of $P(0)$ and of $P(n)\supset P(n+1)$ both consist of a finite number of steps. But as we just saw, the proof technique is justified by the structure of the natural numbers: we know that we can reach any given number eventually, however large, by counting. The “glue” of mathematical induction is the logical connection between the numbers, of which we have \emph{a priori} knowledge.

In the physical world, we do not usually claim we have such knowledge; instead, we rely on the assumption that nature is uniform. The sun has risen every day for as far back as we can remember, so we infer by induction that it will rise tomorrow. One electron behaves like another in any experiment we care to make, so we infer by induction that Ohm’s Law will always hold. But our “induction step”, in the case of science, relies on this assumption of uniformity: there is no “well-founded structure” in the physical world on which we can perform mathematical induction.

In mathematics we can be certain of our results; in physics, if we want to formulate general laws, we are forced to be pragmatic, and take the uniformity of nature on trust. But we have a considerable body of evidence on which to rest our confidence: centuries of careful experimentation combine with millennia of everyday observation to convince us that the universe is, indeed, uniform. Induction is not just, as in Reichenbach’s words, “the best wager we can lay”~\cite{reichenbach1938eap}, it is a very good wager indeed.

\bibliographystyle{plain}
\bibliography{philosophy}

\end{document}

% LocalWords:  Reichenbach’s
words}
\author{Reuben Thomas}
\date{21st October 2021}
\maketitle

Inductive inference is generally held to be an entirely distinct mode of reasoning from mathematical induction. In this essay I shall show that, on the contrary, they are structurally similar, and use this similarity to illuminate the importance of inductive inference to science.

Induction and deduction are methods of argument that use inference to reason from premisses to conclusion.

Deduction applies the rules of some logic to the premisses to reach the conclusion: the reasoning is tautological. In an inductive inference, the conclusion is merely supported by the premisses. Deduction is universally accepted as a valid mode of reasoning.

Induction’s importance to science is that it permits generalization: without it, we would have only knowledge by acquaintance and propositional knowledge, and therefore only an omniscient being could work out general laws. Induction is widely accepted not to be a valid mode of reasoning; the classic statement of “the problem of induction” is given by Hume~\cite{hume1978}. For example, I may observe many white swans, and conclude that all swans are white; but Hume says “the sheer number of impressions has in this case no more effect than if we confined ourselves to one only”: that is, concluding that all swans are white from seeing one white swan is just as valid as reaching the conclusion from seeing a hundred. (And Popper might remind us that it only takes a single black swan to falsify the proposition.)

It is instructive to compare inductive reasoning with mathematical induction. These are typically treated as unrelated modes of reasoning that unfortunately share a name: for example, Wikipedia’s articles on the topics~\cite{wiki:mathematical-induction,wiki:inductive-reasoning} both warn the reader not to confuse the one concept with the other. It is true that mathematical induction is a type of deductive reasoning, but, as I will explain, it is nonetheless a form of induction, and the difference between mathematical induction and inductive reasoning is more subtle than it first appears, and illuminates the importance of inductive reasoning to science.

Mathematical induction is a proof technique used to prove that some proposition $P(n)$ holds for every natural number $0, 1,…$. Mathematical induction can be generalized to structures other than the natural numbers; in general, “structural induction” can be used on any well-founded relation; further details are beyond the scope of this essay. A “proof by induction” consists of two cases: the \textbf{base case}, which proves $P(0)$, and the \textbf{induction step}, which proves that $P(n)$ implies $P(n+1)$. A classic example is the proof that the sum of the first $n$ natural numbers is $\frac{n(n+1)}{2}$.

So, we must prove, for all $n$, $P(n): 0+1+…n = \frac{n(n+1)}{2}$.

First, the base case: $P(0)$ is the proposition $0 = \frac{0(0+1)}{2}$. We can simplify the right-hand side to $\frac{0×1}{2}$, which is $\frac{0}{2}$, which is $0$. So, we have proved the base case.

The induction step is to prove that $P(n)\supset P(n+1)$. In other words, we can assume $P(n)$, and we must prove $P(n+1)$. Substituting $n+1$ for $n$ in the definition of $P(n)$, we see that $P(n+1)$ is the proposition $0+1+…n+(n+1) = \frac{(n+1)(n+2)}{2}$. We can use our assumption of $P(n)$ to substitute $\frac{n(n+1)}{2}$ into the left-hand side, obtaining: $\frac{n(n+1)}{2}+(n+1) = \frac{(n+1)(n+2)}{2}$. Now, we multiply out the right-hand bracket of the right-hand side: $\frac{n(n+1)}{2}+(n+1) = \frac{(n+1)n}{2} + \frac{(n+1)2}{2}$. By rearranging the left-hand fraction and cancelling the $2$s in the right-hand fraction, we obtain: $\frac{n(n+1)}{2}+(n+1) = \frac{n(n+1)}{2} + (n+1)$. We can see that the left-hand side and right-hand side are now identical, so we have proved $P(n+1)$. Hence, “by induction”, we have proved $P(n)$ for all $n$.

To see how this works for all $n$, go back to $P(0)$, which we proved above. In the induction step, we showed that if $P(0)$ is true, then $P(1)$ is true. We can now repeat the induction step: assuming $P(1)$ is true, $P(2)$ must be true; and we can continue to prove $P(n)$ for any $n$, because for any desired $n$ we will reach it eventually by counting up from $0$.

The mathematical argument appears at first to be a finite argument: the proofs of $P(0)$ and of $P(n)\supset P(n+1)$ both consist of a finite number of steps. But as we just saw, the proof technique is justified by the structure of the natural numbers: we know that we can reach any given number eventually, however large, by counting. The “glue” of mathematical induction is the logical connection between the numbers, of which we have \emph{a priori} knowledge.

In the physical world, we do not usually claim we have such knowledge; instead, we rely on the assumption that nature is uniform. The sun has risen every day for as far back as we can remember, so we infer by induction that it will rise tomorrow. One electron behaves like another in any experiment we care to make, so we infer by induction that Ohm’s Law will always hold. But our “induction step”, in the case of science, relies on this assumption of uniformity: there is no “well-founded structure” in the physical world on which we can perform mathematical induction.

In mathematics we can be certain of our results; in physics, if we want to formulate general laws, we are forced to be pragmatic, and take the uniformity of nature on trust. But we have a considerable body of evidence on which to rest our confidence: centuries of careful experimentation combine with millennia of everyday observation to convince us that the universe is, indeed, uniform. Induction is not just, as in Reichenbach’s words, “the best wager we can lay”~\cite{reichenbach1938eap}, it is a very good wager indeed.

\bibliographystyle{plain}
\bibliography{philosophy}

\end{document}

% LocalWords:  Reichenbach’s
